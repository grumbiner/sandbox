\documentstyle[12pt]{article}
\textwidth=6.0in
\textheight=9.0in
\topmargin=-0.5in
\oddsidemargin=0.5in
\renewcommand{\baselinestretch}{2.0}
\newcommand{\grad}{\nabla}
\renewcommand{\div}{\grad \cdot}


\begin{document}
\title{A Theory for Inverting Areal Concentrations to Obtain the Ice Thickness 
Distribution*}
\author{ Robert W. Grumbine \\
National Meteorological Center\\
Camp Springs, MD 20746\\
Donald J. Cavalieri\\
NASA GSFC\\
Greenbelt, MD 20771}
\date {23 October 1995}
\maketitle
\begin{center}
To be submitted to J. Geophysical Research - Oceans  \\
OPC contribution number:
\end{center}

\pagebreak

{\bf Abstract}
    
    The ice thickness distribution is not directly observable on global 
scales.  We present a theory by which the thickness distribution may be
derived given observations of ice concentration, and a knowledge of the
ice dynamics and thermodynamics.  The theory is a variational inversion
scheme, for what turns out to be a wave-like problem.  We find that the
accuracy of the inversion is principally limited by knowledge of ice ridging,
and by the observation of ice cover decay.


\section{Introduction}

    Ice thickness has long been considered an
important feature of the ice pack.  Submarine sonar observations of ice 
thickness extend back to at least [], while airborne laser techniques were 
tested by 1971 [].  Both aerial and submarine measurements are, though,
point measurements of the thickness of an ice floe (the one which is 
in view of the sensor).  This poses something of a conflict with how ice
has tended to be described for theoretical and modelling purposes.  There,
sea ice is treated as a continuum.  Consequently, any region (or grid cell),
must be described by the thicknesses of all the floes in the area, rather
than a single thickness.  A theory for the evolution for that thickness 
distribution was advanced by Thorndike et al. [1975].

    The thickness distribution may be obtained by integrating observations 
of ice (floe) thickness along a length of an observation track.  But, barring
an enormous increase in submarine and aircraft time (and therefore funds), 
these observations will not be global.  And it is the global distribution 
which is required.

    The ice thickness distribution is needed for: correct determination of
air-ice-ocean fluxes of heat and salt, correct forecast and modelling
of ice advance and retreat, and for correct ice-biosphere interactions.  
The heat flux (and, due to the fractioning of salt with respect to ice in
the freezing process, salt) is approximately equal between open water, thin
ice, and thick ice [Maykut, 19??].  This even though the open water and thin
ice constitute only a few percent of the Arctic ice pack.  Ice 
modelling/forecasting requires the thickness for the advance and retreat 
of the ice cover -- thin ice will decay faster, and cannot be advected (as
an ice cover) as easily into warmer waters.  The ice (and snow) thickness
also controls the amount of light which reaches the bottom of the ice floes,
which strongly affects the biological communities which grow there [???].

    The polar ice packs are also expected to be monitor points for climate
change.  If warming is enhanced near the poles, as models predict [????],
the sea ice should get thinner and retreat.  Analysis of ice extent 
observations [Gloersen, Parkinson, Zwally, 199?], have been ambiguous in
defining whether the ice extent has declined in the Arctic.  These analyses
find no decline in Antarctic ice cover.  Submarine observations of ice
thickness [Wadhams, 1993?] have been suggested to indicate that the ice
in the Arctic has thinned.  But this may be due to sampling statistics
rather than an actual change [Walsh, 199?].

    Ice thickness is therefore an intrinsically interesting feature of
the earth system, and has use for climate monitoring.  It is, however,
not something directly observable on global scales.  One feature which
can be observed globally is ice concentration.  If there were a means
of deriving ice thickness distribution from the ice concentration, we
would be able to construct a history of ice thickness back to fall of
1978.  This would presumably be long enough to determine whether the
ice pack has thinned significantly, thickened, or remained approximately
constant.

    We present a theory which will satisfy most of our interests.  The
theory is based on variational data assimilation techniques, which have
a wide use elsewhere [.....].  In our application, we operate under a 
special constraint which is not typical -- the variable of interest is
not observable, but we find compensations -- most noteably that the
propagation of information from better-known areas grows to encompass
the whole domain (in space and time).  What follows is not the only,
or even first, means of responding to this problem.  Kalman filtering
techniques have been applied to the restricted problem of ice type
inferences at limited spatial resolution [Thorndike, Thomas, Rothrock].  
Formally, this could be extended to the problem of the entire thickness
distribution, and at arbitrary spatial resolution.  It is not clear to
us how this should be done.  Nor, given that the Kalman approach assumes
that the model is good, is it obvious that it should be.  The variational
approach that we take is formally equivalent to Kalman filtering in the
limit that the definition of 'good fit' is the same, and the same physical
and observing models are used for both the Kalman and variational method [].
The variational method has also been shown to be formally equivalent to
control theory methods [], and to optimal interpolation [].  Selection from
the various methods, then is a matter of convenience.

    In the next sections, we will describe the observations which we can
use, the governing equations for the evolution of the desired and observable
variables, the theory for extracting the desired information given what is
observable, and finally discuss the merit of what we have when done.  We
will also illustrate the procedure in a theoretical situation, in which 
noisy observations are made on a computed time history of ice thickness 
distribution, and the theory is applied to infer the original history from
those contaminated observations.
\section{Knowables}

    Ice concentration -- Don

    The ice velocity field can be derived from buoy drift observations
augmented by the analyzed winds and pressure field [Colony, 199?] for
the Arctic.  Antarctic coverage of buoys is currently insufficient for
this practice.  The ice velocity field can also be modelled with a 
fairly high correlation between model and observed drift [Flato, Ip, ...]

    The freezing and melting of ice is constrained by the atmosphere and
ocean thermodynamics.  The atmospheric fields are obtainable from analyses
at high (approximately 100 km, 12 hours) resolution.  The oceanic fields are 
more poorly constrained, but forecast experiments with a coupled ice-ocean
model [Hibler and Bryan, 1987], and a coupled ice-mixed layer model [Lemke,
Owens, ..., 198?], have shown that the effects on ice can be modelled
reasonably well.  

    These required fields can be observed or modelled fairly well.  The
fact that they cannot be constrained perfectly requires that we allow for
input errors when developing the theory.
\section{Governing equations}

    The theory for the evolution of the ice thickness distribution was
proposed by Thorndike et al [1975].  Define g(x, y, h, t) to be the 
probability per unit thickness of finding a floe at a location x, y, at 
time t, whose thickness is between h and h+dh.  A thickness h = 0 denotes
open water.  Integrating g over all thicknesses not equal to 0 gives
the probability of finding ice; it is the ice concentration.  The evolution
equation for g is:
\begin{equation}
\frac {\partial g}{\partial t} +
\div \vec{U}g +
\frac {\partial fg} {\partial h} +
= \psi
\end{equation}
\noindent where $\vec{U}$ is the ice velocity field, f is the freezing rate
(thickness per unit time), and $\psi$ describes ridging/rafting/fragmentation.
Physically, the second term says that g may change due to advection of g and
convergence/divergence of g.  This is the normal term for conservation of
a scalar in a compressible fluid.  The third term describes an evolution of
g due to freezing.  If g is large at thickness h, a short time later it has
propagated to thickness h+fdt.  We defer discussion of
ridging until later.  Here, just note that when ice floes collide, they
may crumble at the point of collision and build up significant thickness.
This, of course, affects the probability of finding thick ice, and destroys
some of the thinner ice. The freezing and advection terms have the character 
of a wave equation.  This fact was noted by Coon
et al. [1974], and exploited by Grumbine [1993].  The feature of a wave 
equation is that {\it if} g is known initially, then what occurs is that
the value is carried along characteristics.  Further, if g is always known
at the boundaries, g is known everywhere and for all time.

    To get to an equation with knowable terms, integrate the thickness 
distribution over all thicknesses greater than h0.  This thickness is
not quite zero because until ice has thickened somewhat (5-10 cm) it
is not observable [Cavalieri,...], nor is its growth sensibly described
by the slab approximation [Untersteiner...].  
\begin{eqnarray}
\frac {\partial A}{\partial t} + \div \vec{U}A + fg|^{\infty} - fg|_{h_0}
= \Psi \\
\Psi &=& \int_{h_0}^{\infty} \psi dh \\
A    &=& \int_{h_0}^{\infty} g   dh 
\end{eqnarray}
\noindent  Where we have defined $\Psi$ and A as above.  Equation ?? is now
composed almost entirely of knowable terms.  The evolution of ice 
concentration, A, is observable by satellite.  The ice velocity field is
observable or modellable.  The ice ridging is observable or modellable.
There is no ice infinitely thick, so g, and hence that term, is zero.
The final term is f(h0)g(h0).  The freezing rate of ice at a given thickness
is computable.  

    The only unknown, then is the ice thickness distribution
function for the thinnest ice.  This is precisely the boundary condition which
is needed to evaluate the evolution of g everywhere and for all time.
If we were entirely confident of the knowable terms, we could solve equation
?? for g(h0), and apply it as the boundary condition to equation ??,
which would give the ice thickness distribution for the entire domain of the
problem.  Note, when the ice is freezing, the characteristics advance away
from the thin ice.  i.e., new ice is created and over time it gets thicker.
When the ice is melting, the reverse occurs.  Ice thins, and on reaching
h0, disappears.  A characteristic for an ice floe which forms in the
winter and is destroyed in the summer is illustrated in figure ??.
The observations of ice concentration establish the magnitude of g at
the birth and death of the ice, note by the circles in the figure.  Away from
the observing times, no information is presumed available to constrain
g.  If a submarine happened to pass under an observing cell, this would
provide a constraint for the entire distribution at that time.  The
constraint is noted for the sample ice floe by the x.

\section{Theory}

    There are several ways in which to develop a variational minimization
scheme, and several alternative methods to the variational.  We use the
variational concept, and follow the methods of Sasaki [1976, 1969, 1956].
The strong constraint of Sasaki [19??] represents a constraint which is
known exactly.  This would be, say, conservation of mass in a closed system.
A weak constraint represents a property which should be nearly known,
though some variation is permissable.  We will apply weak constraints to
the requirement that the ice concentration derived by the variational
scheme match the observed.  This should not be a strong constraint as
the observations are imperfect.  Most of our system is represented by
weak constraints.

\section{Application}

    We have implemented computational solutions to evaluate the evolution
of the ice thickness distribution, to emulate noisy observations of ice 
concentration given the ice thickness distribution, and to invert these
noisy observations to obtain an estimated evolution for g(h0), and finally,
given the estimated g(h0), to evaluate the ice thickness distribution from
the inversion method.  We first need to consider some of the computational
requirements for this problem.

    Since the governing equation for ice thickness distribution 
is wave-like, we must satisfy the CFL criterion that the time step be 
sufficiently short that advection does not cross more than 1 cell in a time 
step.  For an ice velocity
of 0.1 m/s and grid spacing of 127 km, this is a time step of 1.27*10$^6$ s,
far longer than anything we'd use given daily observations.  For a freezing
rate of 0.1 m/day (typical for thin ice in winter), and a 0.5 day time step 
(typical of sea ice models), the minimum spacing in thickness is 0.05 m.
Ice freezing rates decline with increasing thickness (due largely to the
effect of insulation provided by thick ice), but even for meter thick ice,
the freezing rate can be 0.03 m/day.  That permits a somewhat coarser grid,
0.15 m.  To model all ice up to 10 meters thick (ice thicker than that is seldom
observed in submarine tracks of the Arctic ocean), we need approximately 
200 thickness levels, spaced 0.05 cm apart.  Ice advection gives no limit,
but ice thermodynamics do.  The number of levels cannot be reduced without
either decreasing the time step (which should be avoided due to the approximately
daily observations, and meteorological forcing which is 12 hourly), or doing
violence to the freezing rate.

    We derive the true solution by evaluating the ice thickness distribution
equation for a specified freezing history.  The boundary condition for thin
ice is set to be a concentration of (1-A)*epsi, where epsi is the fraction
of previously open water which the thinnest ice covers.  Epsi is taken to be
0.1, its exact value being irrelevant to the proceeding.  The freezing is
taken to be ()*sin(omega t).  This is not strictly realistic, as the melting
rate typically has less dependence on thickness than the winter freezing does.
It is, however, unrealistic in a sense which is more difficult to model
or invert.  It is taken for ease of generating a test case.  For the first
test, we set the ice velocity to zero (which also sets ridging to zero).  
Freezing and melting are the most constraining features, occurring as they
do in conjunction with the boundary condition.  

    The noisy observations are emulated by computing, from the ice thickness
distribution, the true ice concentration, and adding a uniformly random
variable, in the range $\pm$ 0.05 in concentration, and then truncating the
observations to concentrations in the range of 0 to 1.  The observations
are taken daily.

    The observations are then fed to the program which solves equations ??,
which provides the new g(bc).  This g(bc) is then fed back to the
forward problem solver as the boundary condition, and the thickness 
distribution computed therefrom.  When all this is done, we arrive at
figures ?? and ??.  Figure ?? presents the true values of ice concentration,
the noisy observations of ice concentration, and the ice concentration after
assimilation.  Figure ?? presents snapshots of the ice thickness distribution
from the true solution and the variational estimate for several different
times in the year.  From this we see clearly that the variational method
is quite good at removing noise from the ice concentration observations.
We also find that the ice thickness distribution can be retrieved with 
good confidence.
\section{Conclusions}

    The most important step, application to real ice observations, including
ice velocity and ridging, is still to be done.  From the above, though,
the project looks like it can indeed be carried out.  The variational
theory developed here can effectively remove noise from the observations
of ice concentration.  Having done so, it can infer the thickness of
the thinnest ice with sufficient accuracy to provide the boundary 
conditions for an ice thickness distribution evolution model.  
\section{Bibliography}
Ketchum

Bourke

Thorndike

Thomas

Rothrock

Thorndike et al.

Flato

Hibler+Bryan

Lemke et al 1987

\end{document}
