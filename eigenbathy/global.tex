\section{Average depth over full scale vs. full resolution eigenperiods}

  We start by taking the 4342 ocean points from Etopo2 along 
30 degrees 1 minute west and finding the full set of eigenvalues for 
purely gravity modes.  While this is not precisely the ocean case, 
gravity modes with relatively little effect of rotation do form much 
of the ocean normal mode spectrum even at periods of hours [Platzman, 
xxxx-xxxx].  So this is a simple and illustrative starting point.

  The gravest mode (numbered 0) has a period of 158,617 seconds (44.06 hours),
5th mode has a period of 28966 seconds (8.05 hours, 8 hours being the fast
limit for Platzman's normal mode evaluations [Platzman, 198x]), 7th mode has 
a period of 22094 seconds (6.14 hours), and 16th is 10911 seconds 
(3.03 hours).  The 100th is 1815 seconds, while 1000th is 190.4 seconds.

  The 4 bathymetry averaging ideas we have are to average H 
(volume-conserving, simplest and certainly most common), average 
sqrt(H) (preserve wave speed, in that shallow water waves have a 
speed of sqrt(gH)), average 1/sqrt(H) (preserve wave travel time 
across a distance), and average 1/H (arrives naturally in considering 
vorticity, which we aren't here, and is symmetric to averaging H in 
the sense that having the pair sqrt(H), 1/sqrt(H) is).

  Figure 1 shows the ratio between averaging method's period and 
full bathymetry's period for each of the modes.  Preserving travel
time is unquestionably the best averaging method in terms of reproducing 
the periods for these modes faster than 3.4 hours.  

  For the slower modes, slower than 3.4 hours, table 1 gives the 
ratios as computed for each averaging method vs. the eigenperiods
as computed:

   Period (s)       H       sqrt(H) 1/sqrt(H)  1/H 
 0 158616.921875  1.025058 1.046656 1.147677 1.305420
  1 86144.515625  0.943714 0.963598 1.056603 1.201828
  2 56781.351562  0.954489 0.974600 1.068667 1.215550
  3 41468.593750  0.980209 1.000862 1.097464 1.248305
  4 35520.730469  0.915475 0.934763 1.024985 1.165865
  5 28965.669922  0.935543 0.955254 1.047454 1.191421
  6 24583.158203  0.944850 0.964757 1.057874 1.203274
  7 22094.478516  0.919867 0.939248 1.029903 1.171458
  8 20616.109375  0.876293 0.894756 0.981117 1.115967
  9 18695.568359  0.869682 0.888005 0.973714 1.107547
 10 17297.593750  0.854517 0.872521 0.956736 1.088235
 11 15683.805664  0.863906 0.882108 0.967248 1.100192
 12 14562.420898  0.858861 0.876956 0.961599 1.093766
 13 13445.416992  0.863769 0.881968 0.967094 1.100017

  Preserving 1/H is always much worse than preserving 1/sqrt(H).
Preserving H is worse than preserving sqrt(H) except for the gravest
mode, where it makes a period error of 2.5\% vs. 4.7\%

  For the slowest modes -- numbered 0-6 here -- preserving sqrt(H)
is typically better than preserving 1/sqrt(H), though by little
on modes 5 and 6, and is worse on mode 4.  Modes 7-13 (hence, 7-1000,
periods shorter than 6.14 hours) are clearly better when preserving 
1/sqrt(H) than sqrt(H).

  On the other hand, if we want the periods to be correct to within,
say, 2\%, it is also clear that no global averaging method is very 
good for the slowest modes exept by chance (sqrt(H) for mode 3 is
very good, 0.1\% relative error, but next best for that method 
is more than 2.5\% off.)  Averaging 1/sqrt(H) provides this degree of
accuracy from mode 14 (12290 seconds) and faster.

  This appears to argue that to obtain good values for the slowest 
gravity modes, it is necessary to permit depth to vary along the
channel.  Given our experience here, we expect that the best averaging
method for this regionally-varying bathymetry will be the 1/sqrt(H). 
We will explore this in the next section.

